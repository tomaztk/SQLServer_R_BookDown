\PassOptionsToPackage{unicode=true}{hyperref} % options for packages loaded elsewhere
\PassOptionsToPackage{hyphens}{url}
%
\documentclass[]{article}
\usepackage{lmodern}
\usepackage{amssymb,amsmath}
\usepackage{ifxetex,ifluatex}
\usepackage{fixltx2e} % provides \textsubscript
\ifnum 0\ifxetex 1\fi\ifluatex 1\fi=0 % if pdftex
  \usepackage[T1]{fontenc}
  \usepackage[utf8]{inputenc}
  \usepackage{textcomp} % provides euro and other symbols
\else % if luatex or xelatex
  \usepackage{unicode-math}
  \defaultfontfeatures{Ligatures=TeX,Scale=MatchLowercase}
\fi
% use upquote if available, for straight quotes in verbatim environments
\IfFileExists{upquote.sty}{\usepackage{upquote}}{}
% use microtype if available
\IfFileExists{microtype.sty}{%
\usepackage[]{microtype}
\UseMicrotypeSet[protrusion]{basicmath} % disable protrusion for tt fonts
}{}
\IfFileExists{parskip.sty}{%
\usepackage{parskip}
}{% else
\setlength{\parindent}{0pt}
\setlength{\parskip}{6pt plus 2pt minus 1pt}
}
\usepackage{hyperref}
\hypersetup{
            pdfborder={0 0 0},
            breaklinks=true}
\urlstyle{same}  % don't use monospace font for urls
\usepackage[margin=1in]{geometry}
\usepackage{longtable,booktabs}
% Fix footnotes in tables (requires footnote package)
\IfFileExists{footnote.sty}{\usepackage{footnote}\makesavenoteenv{longtable}}{}
\usepackage{graphicx,grffile}
\makeatletter
\def\maxwidth{\ifdim\Gin@nat@width>\linewidth\linewidth\else\Gin@nat@width\fi}
\def\maxheight{\ifdim\Gin@nat@height>\textheight\textheight\else\Gin@nat@height\fi}
\makeatother
% Scale images if necessary, so that they will not overflow the page
% margins by default, and it is still possible to overwrite the defaults
% using explicit options in \includegraphics[width, height, ...]{}
\setkeys{Gin}{width=\maxwidth,height=\maxheight,keepaspectratio}
\setlength{\emergencystretch}{3em}  % prevent overfull lines
\providecommand{\tightlist}{%
  \setlength{\itemsep}{0pt}\setlength{\parskip}{0pt}}
\setcounter{secnumdepth}{5}
% Redefines (sub)paragraphs to behave more like sections
\ifx\paragraph\undefined\else
\let\oldparagraph\paragraph
\renewcommand{\paragraph}[1]{\oldparagraph{#1}\mbox{}}
\fi
\ifx\subparagraph\undefined\else
\let\oldsubparagraph\subparagraph
\renewcommand{\subparagraph}[1]{\oldsubparagraph{#1}\mbox{}}
\fi

% set default figure placement to htbp
\makeatletter
\def\fps@figure{htbp}
\makeatother

\usepackage{booktabs}
\usepackage[]{natbib}
\bibliographystyle{plainnat}

\author{}
\date{\vspace{-2.5em}}

\begin{document}

{
\setcounter{tocdepth}{2}
\tableofcontents
}
\hypertarget{hello-bookdown}{%
\section{Hello bookdown}\label{hello-bookdown}}

Hello there!

\hypertarget{this-is-a-table-from-sql-server}{%
\subsection{This is a table from SQL Server}\label{this-is-a-table-from-sql-server}}

\begin{verbatim}
##  Table: \label{tab:table2}SQL Server Example
## |--|--|--|
## | 1 | http://feeds.bbci.co.uk/news/rss.xml | EN |
## | 2 | https://img.rtvslo.si/feeds/00.xml | SL |
\end{verbatim}

\# simple table creation here
\textbar{} Tables \textbar{} Are \textbar{} Cool \textbar{}
\textbar{}---------------\textbar{}:-------------:\textbar{}------:\textbar{}
\textbar{} col 3 is \textbar{} right-aligned \textbar{} \$1600 \textbar{}
\textbar{} col 2 is \textbar{} centered \textbar{} \$12 \textbar{}
\textbar{} zebra stripes \textbar{} are neat \textbar{} \$1 \textbar{}

\begin{longtable}[]{@{}ccc@{}}
\caption{My great data}\tabularnewline
\toprule
\# replace the text below with your table data & ~ & ~\tabularnewline
\midrule
\endfirsthead
\toprule
\# replace the text below with your table data & ~ & ~\tabularnewline
\midrule
\endhead
Tables & Are & Cool\tabularnewline
col 3 is & right-aligned & \$1600\tabularnewline
col 2 is & centered & \$12\tabularnewline
zebra stripes & are neat & \$1\tabularnewline
\bottomrule
\end{longtable}

\hypertarget{a-section}{%
\subsection{A section}\label{a-section}}

A section starts with double hash.

\hypertarget{an-unnumbered-section}{%
\subsubsection*{An unnumbered section}\label{an-unnumbered-section}}
\addcontentsline{toc}{subsubsection}{An unnumbered section}

Chapters and sections are numbered by default. To un-number a heading, add a \texttt{\{.unnumbered\}} or the shorter \texttt{\{-\}} at the end of the heading, like in this section.

\hypertarget{cross}{%
\section{Cross-references}\label{cross}}

This is chpater 2

\hypertarget{chapters-and-sub-chapters}{%
\subsection{Chapters and sub-chapters}\label{chapters-and-sub-chapters}}

and has no headings

\hypertarget{parts}{%
\section{Parts}\label{parts}}

You can add parts to organize one or more book chapters together. Parts can be inserted at the top of an .Rmd file, before the first-level chapter heading in that same file.

Add a numbered part: \texttt{\#\ (PART)\ Act\ one\ \{-\}} (followed by \texttt{\#\ A\ chapter})

Add an unnumbered part: \texttt{\#\ (PART\textbackslash{}*)\ Act\ one\ \{-\}} (followed by \texttt{\#\ A\ chapter})

Add an appendix as a special kind of un-numbered part: \texttt{\#\ (APPENDIX)\ Other\ stuff\ \{-\}} (followed by \texttt{\#\ A\ chapter}). Chapters in an appendix are prepended with letters instead of numbers.

\end{document}
